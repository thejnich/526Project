\documentclass[11pt]{article}   
\usepackage{fullpage}      
\usepackage{color}
\usepackage{hyperref}
\usepackage{cite}

\hypersetup{
    colorlinks,
    citecolor=black,
    filecolor=black,
    linkcolor=blue,
    urlcolor=blue
}

\begin{document}

\title{\bf CPSC 526 - Project Proposal \\ \emph{Public Key Web Authentication}}
\author{Masud Khan, Kyle Milz, Jeff Nicholson}
\date{February 1, 2012} 
\maketitle

\tableofcontents
\pagebreak
\begin{abstract}
Passwords are a necessary part of everyones online activities. They are also a pain. Remembering strong, unique passwords for multiple sites is extremely difficult. This results in users relying on weak passwords, like '123456', which leads to security and privacy concerns. What if there was a way to handle authentication that did not rely on passwords? We hope to provide such an authentication system. We suggest using public key cryptography, which after initial set up, allows users to seamlessly and securely authenticate to web sites, without the need for cumbersome passwords.\\
We will provide a simple authentication protocol, lightweight server side implementation and browser plug in, which can be easily adopted by anyone. A lightweight server side implementation means service providers do not need to make large changes to their implementations. A browser plug-in handles the client side and hides the authentication process from the user, allowing seamless, secure authentication.
\end{abstract}
\pagebreak
\section{Introduction}
Until now, it was almost essential for the average person to remember a large set of username/password combinations for access to some of their most valued belongings and data : Facebook, Twitter, email accounts, cloud storage services, online banking, online shopping, forums, and a plethora of other web applications. Web sites and web based applications need user authentication, to ensure the client is who they say they are. Currently, this authentication is handled by requiring the user to set a username/password pair upon registration with the site. This password should be unique, hard to guess, and only known to the user. Then, in theory, only the user will be able to provide this password and authenticate with the site. In practice, this is not the case. Many users choose very weak passwords \cite{habits}, or reuse passwords across sites \cite{domino}. The few users who do use unique, strong passwords, are then required to remember these passwords, or have their browser store them (in plain text).\\
We propose a new authentication system, which uses public key encryption to handle authentication. After initial set up, authentication is handled with a challenge response between server and client. This will happen behind the scenes and require no user interaction. The system will be more secure than standard passwords. It will not be susceptible to phishing attacks or password leaks due to servers being compromised by an attack.\\
The proposed paper will discuss a secure new method of authentication that will no longer require users to remember even a single password for access to their most valued websites and domains. Section \ref{sec:prob} gives greater detail to the problem we hope to solve. Section \ref{sec:sol} gives an overview of our proposed solution to the problem. Section \ref{sec:deliv} gives an outline of what we hope to accomplish, in terms of concrete software, as well as a loose timeline. Section \ref{sec:relate} briefly cites similar work, and how it differs from our proposed topic.

\section{Problem} \label{sec:prob}
No simpler and more widely scalable method of authentication was ever devised, than to ask a human user to memorize a set of characters and be able to communicate that set, in proper order, at authentication time.  The strength of this method is that the space of possible letter/number combinations is so large that it would be difficult for an attacker to find out any one person's password for any one website/domain, unless the user willingly gave it up.  The weaknesses of this method, however, are that forcing a user to authenticate every time they access a domain is cumbersome at best, and that human memories are not very reliable in comparison to computer memory.  Large scale studies on web password habits have shown that the assortment of passwords most people use is not as secure as they might think \cite{habits}.  Widespread reuse of passwords has limited the long-term viability of the username/password authentication strategy \cite{domino}, and companies are being urged to move on to smart card or biometric authentication systems.  However, these systems come with their own constraints, and are only practical in certain settings. Unlike a password, when a fingerprint pattern is stolen and replicated it cannot be changed, therefore biometrics is only secure when the network and biometric capture devices are secure.  Another security issue is the fact that attacks against passwords and phishing scams have become commonplace, and research into methods to protect users from such threats requires more attention.  A new and widely scalable solution to the problem of authentication can be found, and by piecing together a few different solutions of the past, it can also be implemented quite easily.

\section{Solution} \label{sec:sol}
The high level architecture of the system has three main components, the public key repository/server, the content server, and the client. Other important aspects of the solution are key revocation and use of regular passwords for backups. The protocol for authentication is as follows:
\begin{enumerate}
	\item Client sends an email address and a public key to a key server using, for example, gpg --send-keys.
	\item Client registers for web service using the same email address as provided to the public key server, a password is also chosen at this point to fall back on in case private keys are lost/compromised/not available.
	\item Whenever the client wants to authenticate with the content server, the server sends the email address corresponding to the clients public key to the content server, the content server remotely fetches the public key corresponding to the email address and does a challenge response with the client.
	\item The client decodes the challenge response with their private key and sends back the answer to the server.
	\item The server either accepts or declines the challenge response. If accepted, an authorization token is stored on the client in the form of a cookie, the client uses this cookie whenever it makes requests to the server.
\end{enumerate}

\subsection{Key Repository} \label{sec:sol_key_repository}
This key repository stores $\langle$email address, public key$\rangle$ pairs in a searchable database. The purpose of this is to establish authenticity of public keys using webs of trust. The key server must be secure and resistant against attacks. PGP key servers offer these features and will be used for this project, as implementing and maintaining a key server is out of the scope of this project. \\
Keys can be submitted via gpg --send-keys. This dependency is heavy weight and not all people will want this extra hassle, however man-in-the-middle attacks are quite easy if webs of trust cannot be used. \\

\subsection{Content Server} \label{sec:sol_server}
Minimizing the amount of code that needs to be on the server side was an important design constraint. The barrier for deployment of server side code is directly proportional to the amount of changes. \\
Upon receiving an authentication request, typically an email address, the content server makes a remote request to the key server for the email addresses' corresponding public key. Upon obtaining the public key, the server generates a random byte string and encodes it with the public key. This is then sent to the client for decoding. \\
Upon decoding, the client sends a hash of the decrypted random bytes. The server checks that this hash in fact matches a hash of the original random bytes it generated and sent. If the check is successful, an authentication cookie is given to the client which the client uses in each authenticated request. If the check is unsuccessful, authentication is denied. \\
The content server then keeps track of the authentication tokens it has issued and provides content based on them.

\subsection{Client/Browser Extension} \label{sec:sol_client}
In this scheme the client does most of the work, which is important in mitigating denial of service attacks.  \\
The client is responsible for sending authentication requests to the content server, typically an email address associated with their public keys. The content server then does a challenge response with the client. \\
Two possible behaviours are going to be investigated for the clients half of the challenge response. \\
\begin{enumerate}
	\item Make an ssh-agent/equivalent extension for the browser to allow no dialog decryption using ssh private key.
	\item Use a dialog box for the user to enter their private key password, or alternatively if using a password-less private key then no dialog box would be needed.
\end{enumerate}

\subsection{Key Revocation} \label{sec:sol_key_revo}
When using public/private keys one must always consider what happens if a private key is lost or compromised. A key revocation mechanism is already part of the PGP key server. 

\subsection{Passwords} \label{sec:sol_pw}
The system will use password based authentication in addition to the new authentication scheme that is proposed. This is used in case a private key is unavailable like the user being at a public computer instead of their own private one.


\section{Deliverables and Timeline} \label{sec:deliv}
The timeline will be split into roughly two equal sections, client side and server side modifications, each taking roughly one month to complete:
\begin{itemize}
	\item A single client side public key implementation, as described in Section \ref{sec:sol_client},  must be done, most likely in Java-script. Browser extensions may have to be written for seamless login behaviours.
	\item A server side implementation, as described in Section \ref{sec:sol_server},  must be done per web service that wants to enable this authentication system. That includes public key retrieval, and a public key implementation. Languages used here include perl, ruby, python, asp, cgi, etc.
\end{itemize}

The aim is to port the new authentication system to as many back-ends as possible to ease integration but because of manpower and time constrains at least one back-end will be implemented. If time permits more back-ends will be coded.

\section{Related Works} \label{sec:relate}
The topic of web authentication has been a hot one for some time, which is why it's no surprise that there are numerous other projects that have tried to create different solutions.  Among these other projects is a project called pwdhash at Stanford \cite{pwdhash}.  Pwdhash uses a pair of pieces of information (password, domain-name) that is hashed by a publicly available hash function which can be computed on any computer.  The hash is then sent as the password to the website.  With this setup, the user is protected from phishing websites that pretend to be another domain, and the user's password is not sent over the connection – just a hash of it.  Therefore an eavesdropper cannot steal the password and gain access to all sites that the user has re-used that password.  Pwdhash is available as a browser plug-in.\\
Public key authentication has also been around for a while, the most reputed implementation of which is ssh.  It works by storing public and private keys in the different party's drives, and when the user attempts a connection to a remote server, public-key encryption is used in each exchange of information.  This stops an attacker from seeing the information in transit, though the attacker could still change the information before it gets to it's destination.\\
Challenge response authentication is a very broad topic.  The most basic type of challenge response would be a server requesting a password from the user in order to authenticate.  More involved extensions of challenge response include zero-knowledge password proof (ZKPP), a procedure where only enough data is exchanged between one party (the prover) and another party (the verifier) to let the verifier know that the prover knows the password.  In ZKPP, nothing is revealed except the fact that the prover knows the password.  Our implementation uses a challenge response as a part of authentication.  There are a plethora of different challenge response methods, and our use of the idea is only a single part of this paper's scheme.

\bibliographystyle{plain}
\bibliography{refs}{}
\addcontentsline{toc}{section}{References}

\end{document}
