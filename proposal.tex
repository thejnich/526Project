\documentclass[11pt]{article}   % list options between brackets
\usepackage{fullpage}              % list packages between braces
\usepackage{color}
\usepackage{hyperref}
\usepackage{cite}

\hypersetup{
    colorlinks,
    citecolor=black,
    filecolor=black,
    linkcolor=blue,
    urlcolor=blue
}

\begin{document}

\title{\bf CPSC 526 - Project Proposal \\ \emph{catchy smart sounding title TBD}}   % type title between braces
\author{Masud Kahn, Kyle Milz, Jeff Nicholson}         % type author(s) between braces
\date{February 1, 2012}    % type date between braces
\maketitle

\tableofcontents
\pagebreak
\begin{abstract}
some abstract statements about web security and authentication and how we are going to solve all the worlds problems, including famine.
\end{abstract}
\pagebreak
\section{Introduction}
Introduce problem we are trying to solve (passwords, secure auth etc.). Brief intro of our proposed solution

\section{Problem}
Until now, it was almost essential for the average person to remember a large set of username/password combinations for access to some of their most valued belongings & data : facebook, twitter, email accounts, cloud storage services, online banking, online shopping, forums, and a plethora of other web applications.  No simpler and more widely scalable method of authentication was ever devised, than to ask a human user to memorize a set of characters and be able to communicate that set, in proper order, at authentication time.  The strength of this method is that the space of possible letter/number combinations is so large that it would be difficult for an attacker to find out any one person's password for any one website/domain, unless the user willingly gave it up.  The weaknesses of this method, however, are that forcing a user to authenticate everytime they access a domain is cumbersome at best, and that human memories are not very reliable in comparison to computer memory.  Large scale studies on web password habits have shown that the assortment of passwords most people use is not as secure as they might think[web password habits reference].  Widespread reuse of passwords has limited the long-term viability of the username/password authentication strategy[password reuse paper reference], and companies are being urged to move on to smart card or biometric authentication systems.  However, these systems come with their own constraints, and are only practical in certain settings – unlike a password, when a fingerprint pattern is stolen and replicated it cannot be changed, therefore biometrics is only secure when the network and biometric capture devices are secure.  A new and widely scalable solution to the problem of authentication can be found, and by piecing together a few different solutions of past, it can also be implemented quite easily.  This paper will discuss a secure new method of authentication that will no longer require users to remember even a single password for access to their most valued websites and domains.

\section{Solution}
Detailed explanation of our proposed solution. Subsection for each part maybe (client/browser extension, server, central key repo).
Mention method for revoking a key in the central repo, so that if a users private key is compromised they can protect themselves somewhat. This may already be implemented, i dunno.

\section{Deliverables and Timeline}
Specific deliverables and timeline for each

\section{Related Works}
Maybe discuss similar existing tech. pwdhash, ssh public key auth.\\
Some papers that we can maybe reference in proposal:
\begin{itemize}
\item domino effect of pwd reuse \cite{domino}
\item study of password habits \cite{habits}
\end{itemize}

\bibliographystyle{plain}
\bibliography{refs}{}

\end{document}
