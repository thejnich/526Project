\documentclass[11pt]{article}
\usepackage{fullpage}
\usepackage{color}
\usepackage{hyperref}
\usepackage{cite}

\hypersetup{
    colorlinks,
    citecolor=black,
    filecolor=black,
    linkcolor=blue,
    urlcolor=blue
}

\begin{document}

\title{\bf CPSC 526 - Status Update \\ \emph{Public Key Web Authentication}}
\author{Masud Khan, Kyle Milz, Jeff Nicholson}
\date{February 28, 2012}
\maketitle

\tableofcontents
\pagebreak

\section{Project Progress} \label{sec:progress}
In working towards our goal of implementing a public-key based authorization system for web domains, we have come across some useful projects done by others (Section \ref{sec:relatedWorks}), some of which have somewhat working implementations.  Although we will be using code from these projects, none are implemented with the same specifications that we've decided on, so some further tweaking is necessary.  For example, \nameref{subsec:gpgauth} uses a server-side database to store public keys of registered users, whereas we'd like our implementation to query a public key server given the user's email address, to alleviate the need for a cumbersome database which must be maintained, protected and kept up-to-date. We only require a database to maintain state for users who have requested authentication, in the firt step of our 3-way handshake.\\
Currently we are in the process of setting up dependencies and configurations in order to get the different pieces of our implementation working together.  We are currently working on four different server implementations, running on \nameref{subsec:php}, \nameref{subsec:perl}, \nameref{subsec:ruby} and \nameref{subsec:python}. Our code is available publicly \cite{526proj}.\\

\subsection{PHP} \label{subsec:php}
The php version is forked from the \nameref{subsec:gpgauth} project, and works in conjunction with the gpgAuth Chrome browser extension.  Though we've got these tools working, we are still in the process of learning how they work so that we can change the implementation to fit our specifications. 
\subsection{Perl} \label{subsec:perl}
The perl implementation is running and can detect a user's input. There are many comprehensive perl gpg libraries. They allow for creation of key pairs, and querying of key servers.
\subsection{Ruby} \label{subsec:ruby}
The ruby implementation is running on top of the rails framework and is currently able to query the ubuntu keyserver and retrieve a public key given a valid e-mail address. 
\subsection{Python} \label{subsec:python}
The python implementation is in it's initial stages. 

\section{Related Works} \label{sec:relatedWorks}
\subsection{Pwdhash} 
Pwdhash\cite{pwdhash} is a project focused on creating an easy way to make the traditional web authentication protocol more secure. It uses a publicly available hash function to hash the domain name and the users password. The hash is then sent as the password to the website.\\
	With this setup, the user is protected from phishing websites that pretend to be another domain, and the user’s password is not sent over the connection - just a hash of the user's password. Therefore an eavesdropper cannot steal the password and gain access to all sites that the user has re-used that password.\\
	On the downside, the user still has to remember a password, though different browser plug-ins can be configured to alleviate this caveat.\\
The benefit of this method, is that it is purely client side. There is no implementation needed on the server. They still store usernames and passwords, and authenticate as before.

\subsection{Zero-knowledge password proof example} 
Zero-knowledge password proof (ZKPP)\cite{huth}, a procedure where only enough data is exchanged between one party (the prover) and another party (the verifier) to let the verifier know that the prover knows the password.  In ZKPP, nothing is revealed except the fact that the prover knows the password.  At first this seems impossible, but with some number theory, it becomes simple\cite{gqprotocol}.\\
A slightly simpler version of this type of authentication is described by Huth\cite{huth}. In this scheme, a trusted authority must provide the prover (client) with her own personal identity information, $x[0]$, and a random number $k$.  The trusted authority provides the verifier (server) with $x[k]$ and $k$, where $x[k]$ is the value $x[0]$ hashed $k$ times, using a hash function, $hash()$.\\
	The verification of a prover's identity then consists of a successful run of the following protocol.  The prover sends the verifier her personal identity information $x[k]$.  The verifier matches that with his stored value $x[k]$.  This corresponds to asking for a login name - a mismatch results in a failed login attempt.  The verifier asks for a password, which is any value $x\prime$ such that $hash(x\prime)=x[k]$. Since the prover knows the secret $x[0]$, she can use $hash()$ to compute $x[k-1]$ as such an $x\prime$ and send this to the verifier.  The verifier can compute $hash(x[k-1])$ and match it with $x[k]$.  After a successful match, the verifier grants access to the prover, stores $x[k-1]$ as the current identity information of that user, and the next authentication round will be done with $x[k-1]$ and $x[k-2]$ instead of $x[k]$ and $x[k-1]$.\\
	The benefits of this protocol are that no secret information is passed between the client and server.  If an eavesdropper were listening to the communication between the client and server, they may detect $x[k]$, but since they don't know the current value of $k$ and the hash function, they will not be able to calculate the proper values needed for the next authentication attempt.\\
	The downside is the vulnerability of users to phishing attacks, and also of replay attacks.  If the server goes down right after the prover provided $x[k-1]$ along an insecure channel, an eavesdropper could, upon recovery of the verifier, replay $x[k]$, provided that the verifier would remember $x[k]$ as the current password of the alleged user.\\
In this simple protocal, the prover had to provide their identity to the verifier, in the form of $x[k]$. By applying a slightly more complex protocol, like the Guillou-Quisquater protoclol\cite{gqprotocol}, you can achieve a zero-knowledge proof protocol.

\subsection{Pacheco Paper}
David Pacheo describes public key authentication in his paper \cite{pacheco}. His proposed method of public key authentication is very similar to our protocol. He gives an overview of the reasons for needing a change in web authentication, and the main features needed to be implemented. \\
The general idea as described by Pacheco is as follows: When a user wants to enter a website, he enters his email address into a form. Upon verifying that the given email address corresponds to a real user of the web site, the web server then queries the central key authority for the user's public key. Using challenge-response authentication, the server can verify the identity of the user.\\
	This design ensures that the server-side can be sure that the user is who she says she is, but the client-side is still susceptible to phishing attacks, since there is no challenge-response from the client to the server using a public key from the server.  This security hole could lead to the user on the client-side being tricked into providing sensitive information to a phishing website.\\
	On the other hand, the design uses a central key authority to store keys of users, which means less implementation is required on the server-side, and the function of protecting the integrity of users' keys is encapsulated away from both the user and the domain.  Having key-handling abstracted out of the design creates a more modular and re-useable implementation.\\

\subsection{GPGAuth} \label{subsec:gpgauth}
GpgAuth\cite{gpgauth} is an authentication mechanism which uses Public/Private (cryptographic) keys (such as GnuPG, PGP) to authenticate users to a web-page/service.  The process works through the two-way exchange of encrypted and signed tokens between a user and the service.\\
	The process works as follows: upon entering a website, the client uses the server's public key to encrypt a token of random data, generated locally.  The encrypted token is sent to the server, where it is decrypted using the server's private key, and then signed by the server before being returned.  The server then generates it's own token of random data and encrypts it to the tune of the client's public key.  The client's decrypted and signed token  and the server's encrypted token are then sent back to the client, who then knows that the server is who it says it is.  The client uses it's private key to decrypt the server's token, sign it, and then return it to the server, who then knows that the client is who it says it is.\\
	In this implementation, a server-side database is required in order to store the public keys of the users, and the client is required to remember the public keys of domains/web services. It provides symmetric authentication. This is optional, you can easily leave out the server authentication, and just have the client authenticate to the server. It depends on the application, and how important it is that the client trust the identity of the server.\\
	This project is very similar to our implementation, and parts of the actual implementation will be used by us.

\section{Moving Forward} \label{sec:future}
The next steps are to reduce the dependency on a server-side database in the php implementation by connecting gpgAuth to a key-server query tool.  When we're able to do this, we can move on to porting the php implementation to perl and ruby, time allowing.  

\subsection{PHP} \label{subsec:php}
\bibliographystyle{plain}
\bibliography{refs}{}
\addcontentsline{toc}{section}{References}

\end{document}

